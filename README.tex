% Created 2015-04-08 Wed 15:51
\documentclass[11pt]{article}
\usepackage[utf8]{inputenc}
\usepackage[T1]{fontenc}
\usepackage{fixltx2e}
\usepackage{graphicx}
\usepackage{longtable}
\usepackage{float}
\usepackage{wrapfig}
\usepackage{rotating}
\usepackage[normalem]{ulem}
\usepackage{amsmath}
\usepackage{textcomp}
\usepackage{marvosym}
\usepackage{wasysym}
\usepackage{amssymb}
\usepackage{hyperref}
\tolerance=1000
\usepackage[english]{babel}
\usepackage[left=2.5cm,right=2.5cm,top=2.8cm,bottom=3cm]{geometry}
\setlength{\parindent}{0cm}
\setlength{\parskip}{0.1cm}
\usepackage{pmboxdraw}
\DeclareUnicodeCharacter{00A0}{~}
\DeclareUnicodeCharacter{251C}{\pmboxdrawuni{251C}}
\renewcommand{\arraystretch}{1.2}
\author{Tim Krones}
\date{\today}
\title{ODE README}
\hypersetup{
  pdfkeywords={},
  pdfsubject={},
  pdfcreator={Emacs 24.4.1 (Org mode 8.2.10)}}
\begin{document}

\maketitle
\tableofcontents


\section{About}
\label{sec-1}
ODE (Output DEsigner) is a system for rapid development of
large-scale rule bases for template-based natural language
generation for conversational agents.

It was developed in the context of \href{http://www.aliz-e.org/}{ALIZ-E}, a \href{http://www.dfki.de/lt/project.php?id=Project_576&l=en}{project} that was
carried out jointly by the German Research Center for Artificial
Intelligence (\href{http://www.dfki.de/}{DFKI}) and a number of European partners.

\section{Setting up}
\label{sec-2}
\subsection{Dependencies}
\label{sec-2-1}
ODE requires:

\begin{itemize}
\item \textbf{Java SDK 7} (Oracle JDK 7 or OpenJDK 7).
It has not been tested with Java 8, although newer versions of
play! (the MVC framework on which ODE is built) \href{https://github.com/playframework/playframework/issues/1220}{seem to be Java 8-ready}.
\item \textbf{play-2.2.1}
\item \textbf{neo4j-community-2.1.2} (bundled with this distribution of ODE).
\item \textbf{Twitter Bootstrap v3.0.2} (\href{https://en.wikipedia.org/wiki/Minification_(programming)}{minified} version bundled with this
distribution of ODE).
\item \textbf{jQuery v1.11.0} (minified version bundled with this distribution
of ODE).
\item \textbf{jQuery UI v1.10.4} (minified version bundled with this
distribution of ODE).
\item \textbf{Underscore.js v1.5.2} (minified version bundled with this
distribution of ODE).
\item \textbf{Backbone.js v1.1.0} (minified version bundled with this
distribution of ODE).
\end{itemize}

\begin{center}
\begin{tabular}{lrll}
\hline
\textbf{Dependency} & \textbf{Version} & \textbf{Bundled} & \textbf{Setup required}\\
\hline
Java SDK & 7 & no & yes\\
play! & 2.2.1 & no & yes\\
Neo4j & 2.1.2 & yes & yes\\
Twitter Bootstrap & 3.0.2 & yes (minified) & no\\
jQuery & 1.11.0 & yes (minified) & no\\
jQuery UI & 1.10.4 & yes (minified) & no\\
Underscore.js & 1.5.2 & yes (minified) & no\\
Backbone.js & 1.1.0 & yes (minified) & no\\
\hline
\end{tabular}
\end{center}

\subsection{Initial setup}
\label{sec-2-2}
Follow these steps to prepare your environment for working on ODE:

\begin{enumerate}
\item Install \textbf{Java SDK 7}.

\item If you haven't done so already, clone the ODE repository from
GitHub:

\begin{verbatim}
git clone https://github.com/itsjeyd/ODE
\end{verbatim}

\item Extract \url{deps/neo4j-community-2.1.2-unix.tar.gz}.

\item Load initial data:

\begin{verbatim}
cd /path/to/neo4j-community-2.1.2/
bin/neo4j-shell -path data/graph.db/ -file /path/to/this/repo/initial-data.cql
\end{verbatim}

The output of the second command should look like this:

\begin{verbatim}
+-----------------------------------------------------+
| n                                                   |
+-----------------------------------------------------+
| Node[0]{username:"dev@ode.com",password:"password"} |
+-----------------------------------------------------+
1 row
Nodes created: 1
Properties set: 2
Labels added: 1
1880 ms

+--------------------------------+
| n                              |
+--------------------------------+
| Node[1]{name:"underspecified"} |
+--------------------------------+
1 row
Nodes created: 1
Properties set: 1
Labels added: 1
36 ms
\end{verbatim}

\item Download play! (Version 2.2.1) from \href{http://downloads.typesafe.com/play/2.2.1/play-2.2.1.zip}{here} and extract it. Make
sure you choose a location to which you have both read and write
access.

\item Make sure that the \texttt{play} script is executable:

\begin{verbatim}
cd /path/to/play-2.2.1/
chmod a+x play
chmod a+x framework/build
\end{verbatim}

\item Add directory of \texttt{play} executable to your \texttt{PATH}:

\begin{verbatim}
export PATH=$PATH:/path/to/play-2.2.1/
\end{verbatim}

Add this code to your \texttt{.bashrc}, \texttt{.zshrc}, etc. to make the
modification permanent.
\end{enumerate}

\subsection{Daily workflow}
\label{sec-2-3}
\subsubsection{Before}
\label{sec-2-3-1}
\begin{enumerate}
\item Start Neo4j:

\begin{verbatim}
cd /path/to/neo4j-community-2.1.2/
bin/neo4j start-no-wait
\end{verbatim}

\item Start play!:

\begin{verbatim}
cd /path/to/this/repo/ode/
play
\end{verbatim}

\item Run application (from \texttt{play} console):

\begin{verbatim}
run
\end{verbatim}

\item Access application by navigating to \url{http://localhost:9000/} in
your browser.

When you do this for the first time you will also need to perform
the following steps:

\begin{enumerate}
\item Click the ``Login'' button in the top-right corner

\item Enter credentials:
\begin{itemize}
\item Email: \texttt{dev@ode.com}
\item Password: \texttt{password}
\end{itemize}
\end{enumerate}
As you make changes to the code, refreshing the current page in
the browser will cause \texttt{play} to recompile the project.
Note that compilation will only be triggered after changes to
(server-side) code that actually \emph{needs} to be compiled.
Modifying client-side code will not trigger compilation.
\end{enumerate}

\subsubsection{After}
\label{sec-2-3-2}
\begin{enumerate}
\item Stop application (from \texttt{play} console): \texttt{Ctrl-D}

\item Stop Neo4j:

\begin{verbatim}
cd /path/to/neo4j-community-2.1.2/
bin/neo4j stop
\end{verbatim}
\end{enumerate}

\subsubsection{Accessing Neo4j directly}
\label{sec-2-3-3}
You can access the graph database directly by navigating to
\url{http://localhost:7474/browser/} in your browser. This gives you a
graphical interface for entering Cypher commands to interact with
the database.

Neo4j also comes with a command line interface (``Neo4j Shell'') for
interacting with databases. After stopping the database as
described above you can issue the following command to start the
shell:

\begin{verbatim}
bin/neo4j-shell -path data/graph.db/
\end{verbatim}

More information about how to work with the Neo4j Shell can be
found \href{http://neo4j.com/docs/2.1.2/shell.html}{here}.

\section{Project Structure}
\label{sec-3}
\begin{verbatim}
.
├── .git
├── deps
├── doxygen
├── ode
├── training-materials
├── initial-data.cql
├── README.org
├── README.pdf
└── README.tex
\end{verbatim}

\subsection{\texttt{deps}}
\label{sec-3-1}
\begin{verbatim}
deps/
└── neo4j-community-2.1.2-unix.tar.gz
\end{verbatim}

This folder contains third-party software that ODE depends on.

\subsection{\texttt{doxygen}}
\label{sec-3-2}
\begin{verbatim}
doxygen/
├── Doxyfile
└── html.tar.gz
\end{verbatim}

This folder contains documentation for server-side code in HTML
format. After extracting \href{doxygen/html.tar.gz}{html.tar.gz}, the entry point for viewing
the documentation is \href{doxygen/html/index.html}{html/index.html}. A graphical representation of
the class hierarchy is available under \href{doxygen/html/inherits.html}{html/inherits.html}.

To regenerate the documentation after modifying the source code,
run the following commands:

\begin{verbatim}
cd /path/to/this/repo/doxygen
doxygen Doxyfile
\end{verbatim}

Note that this will overwrite the contents of the \texttt{html} folder
that contains the documentation extracted from \texttt{html.tar.gz}.

On many Linux distributions, Doxygen can be installed from
official package repositories. It can also be built from source on
Unix and Windows as described \href{http://www.stack.nl/~dimitri/doxygen/manual/install.html}{here}. The documentation bundled with
this distribution of ODE was generated using Doxygen 1.8.9.

\subsection{\texttt{ode}}
\label{sec-3-3}
\begin{verbatim}
ode/
├── app
├── conf
├── logs
├── project
├── public
├── test
├── build.sbt
└── README
\end{verbatim}

This folder contains the complete source code of ODE. While
extending ODE you will mostly be working with files located in the
\texttt{app}, \texttt{public}, and \texttt{test} directories.

\subsubsection{\texttt{app}}
\label{sec-3-3-1}
\begin{verbatim}
ode/app/
├── constants
├── controllers
├── managers
├── models
├── neo4play
├── utils
├── views
└── Global.java
\end{verbatim}

\paragraph{\texttt{constants}:}
\label{sec-3-3-1-1}
Enums that define \textbf{node and relationship types}.

\paragraph{\texttt{controllers}:}
\label{sec-3-3-1-2}
Classes that implement \textbf{methods for handling user requests}. Each
controller method is associated with a specific type of HTTP
request (\texttt{GET}, \texttt{POST}, \texttt{PUT}, \texttt{DELETE}) and URL (cf. Section
\ref{sec-3-3-2} below).

\paragraph{\texttt{managers}:}
\label{sec-3-3-1-3}
Classes that \textbf{handle communication with the \hyperref[sec-3-3-1-5]{database access layer}}.

Each \hyperref[sec-3-3-1-4]{model class} has a static \texttt{nodes} field or a static
\texttt{relationships} field that stores a reference to an appropriate
Manager object. Managers \textbf{implement appropriate CRUD (Create,
Read, Update, Delete) methods} for obtaining and operating on
model data. When handling user requests, \hyperref[sec-3-3-1-2]{controllers} call these
methods via the \texttt{nodes} and \texttt{relationships} fields of relevant
model classes.

\paragraph{\texttt{models}:}
\label{sec-3-3-1-4}
\textbf{Classes representing domain entities} (such as rules, features,
and values) \textbf{and relationships} between them.

\paragraph{\texttt{neo4play}:}
\label{sec-3-3-1-5}
Classes that implement a custom \textbf{database access layer} for
communicating with Neo4j.

\paragraph{\texttt{utils}:}
\label{sec-3-3-1-6}
\textbf{Utility classes} for manipulating strings and generating Version
3 and Version 4 UUIDs. Any additional utility classes that you
implement should be added to this folder.

\paragraph{\texttt{views}:}
\label{sec-3-3-1-7}
\textbf{Server-side templates} for rendering different user interfaces.
Controllers will inject relevant data into these templates when
users request corresponding interfaces. Note that most rendering
operations actually happen on the client; the templates in this
folder only provide basic scaffolding for the different
interfaces that are part of the current implementation.

\paragraph{\texttt{Global.java}:}
\label{sec-3-3-1-8}
Defines \textbf{custom global settings} for ODE. Currently, the \texttt{Global}
class defines how ODE should behave for different types of
errors.

\subsubsection{\texttt{conf}}
\label{sec-3-3-2}
\begin{verbatim}
ode/conf/
├── application.conf
└── routes
\end{verbatim}

This folder contains configuration files for ODE.

\texttt{application.conf} is the main configuration file; it contains
standard configuration parameters. You should not have to touch
this file very often during day-to-day development.

\texttt{routes} defines mappings between pairs of the form \texttt{<HTTP-verb>
    <URL>} and \hyperref[sec-3-3-1-2]{controller methods}:

\begin{verbatim}
# Home page
GET     /                           controllers.Application.home()

# Authentication
POST    /register                   controllers.Auth.register()
GET     /login                      controllers.Application.login()
POST    /login                      controllers.Auth.authenticate()
GET     /logout                     controllers.Application.logout()

# Browsing
GET     /rules                      controllers.Rules.browse()
GET     /rules/:name                controllers.Rules.details(name: String)

...
\end{verbatim}

Every time you add a new controller method that renders an
additional interface or serves as an endpoint for AJAX requests,
you have to define a URL for it in this file.

\subsubsection{\texttt{logs}}
\label{sec-3-3-3}
\begin{verbatim}
ode/logs/
└── application.log
\end{verbatim}

This folder contains log files produced by ODE. By default, all
logging output is written to \texttt{application.log}.

\subsubsection{\texttt{project}}
\label{sec-3-3-4}
\begin{verbatim}
ode/project/
├── build.properties
└── plugins.sbt
\end{verbatim}

play! applications are built using \href{http://www.scala-sbt.org/}{sbt} (Scala Build Tool). This
folder contains the \texttt{sbt} build definitions; \texttt{plugins.sbt} defines
\texttt{sbt} plugins used by ODE, and \texttt{build.properties} contains the
\texttt{sbt} version to use for building the application.

You should not have to touch the files in this folder very often
during day-to-day development.

\subsubsection{\texttt{public}}
\label{sec-3-3-5}
\begin{verbatim}
ode/public/
├── css
│   ├── lib
│   │   └── bootstrap.min.css
│   ├── browse.css
│   ├── details.css
│   ├── features.css
│   ├── input.css
│   ├── main.css
│   ├── output.css
│   └── search.css
├── fonts
│   └── ...
├── images
│   └── ...
└── js
    ├── lib
    │   ├── backbone-min.js
    │   ├── bootstrap.min.js
    │   ├── jquery.min.js
    │   ├── jquery-ui.min.js
    │   └── underscore-min.js
    ├── browse.js
    ├── combinations.js
    ├── details.js
    ├── error.js
    ├── features.js
    ├── header.js
    ├── input.js
    ├── ode.js
    ├── output.js
    └── search.js
\end{verbatim}

This folder contains code that implements client-side
functionality of ODE. The following table shows associations
between routes, controller methods, server-side templates, and
corresponding client-side code (CSS and JavaScript):

\begin{center}
\begin{tabular}{lllll}
\hline
\textbf{Route} & \textbf{Controller} & \textbf{Template} & \textbf{CSS} & \textbf{JS}\\
\hline
\texttt{GET /} & \texttt{Application.home} & home.scala.html & - & -\\
\texttt{GET /rules} & \texttt{Rules.browse} & browse.scala.html & browse.css & browse.js\\
\texttt{GET /rules/:name} & \texttt{Rules.details} & details.scala.html & details.css & details.js\\
\texttt{GET /features} & \texttt{Features.features} & features.scala.html & features.css & features.js\\
\texttt{GET /rules/:name/input} & \texttt{Rules.input} & input.scala.html & input.css & input.js\\
\texttt{GET /rules/:name/output} & \texttt{Rules.output} & output.scala.html & output.css & output.js\\
\texttt{GET /search} & \texttt{Search.search} & search.scala.html & search.css & search.js\\
\hline
\end{tabular}
\end{center}

Each of the JS modules listed above makes use of a number of
utility functions for

\begin{itemize}
\item operating on strings
\item creating new DOM elements
\item operating on existing DOM elements.
\end{itemize}

These functions are defined in the \texttt{ode.js} module.

\subsubsection{\texttt{test}}
\label{sec-3-3-6}
\begin{verbatim}
ode/test/
├── controllers
├── managers
├── models
├── neo4play
├── utils
├── views
└── IntegrationTest.java
\end{verbatim}

This folder contains tests for server-side functionality of ODE.
Its structure parallels the structure of the \texttt{app} folder: Tests
for controllers are located in the \texttt{controllers} folder, tests for
utilities are located in the \texttt{utils} folder, etc.

To run the test suite:

\begin{verbatim}
cd /path/to/this/repo/ode/
play test
\end{verbatim}

You can also run the tests from the \texttt{play} console. The sequence
of commands then becomes:

\begin{verbatim}
cd /path/to/this/repo/ode/
play
test
\end{verbatim}

\subsubsection{\texttt{build.sbt}}
\label{sec-3-3-7}
This file contains the main build declarations for ODE.

\subsection{\texttt{training-materials}}
\label{sec-3-4}
\begin{verbatim}
training-materials/
├── assignment
│   ├── assignment.html
│   ├── gold-standard
│   │   ├── input.org
│   │   └── output.org
│   ├── reference.pdf
│   └── reference.tex
├── css
│   └── ...
├── js
│   └── ...
├── questionaire
│   ├── answers
│   │   └── data.json
│   ├── persist.php
│   └── questionaire.html
├── sounds
│   └── ...
├── training
│   ├── 00-intro.html
│   ├── 01-create_rule.html
│   ├── 02-add_feature.html
│   ├── 03-set_value.html
│   ├── 04-remove_feature.html
│   ├── 05-rename_rule.html
│   ├── 06-change_description.html
│   ├── 07-switching.html
│   ├── 08-add_output_string.html
│   ├── 09-modify_output_string.html
│   ├── 10-remove_output_string.html
│   ├── 11-split_output_string.html
│   ├── 12-add_part.html
│   ├── 13-show_output.html
│   ├── 14-modify_part.html
│   ├── 15-remove_part.html
│   ├── 16-add_slot.html
│   ├── 17-remove_slot.html
│   ├── 18-parts_inventory.html
│   ├── 19-multiple_groups.html
│   └── 20-browse_rules.html
├── intermission.html
└── overview.html
\end{verbatim}

This folder contains materials that can be used to train novice
users to use ODE (and to gather feedback about the system). The
entry point for starting the training process is \href{training-materials/overview.html}{overview.html}.

Feedback submitted via the questionnaire will be stored in
\href{training-materials/questionaire/answers/data.json}{data.json}. \textbf{Note} that in order for this to work,

\begin{enumerate}
\item a web server (such as Apache) has to serve the file \href{training-materials/questionaire/persist.php}{persist.php}
      at \url{http://localhost/persist.php}
\item the user under which the web server is running must have write
access to \href{training-materials/questionaire/answers/data.json}{data.json}.
\end{enumerate}

Additionally, as a preparatory step the string
\texttt{/path/to/this/repo/} in line 8 of \texttt{persist.php} has to be
replaced with the absolute path of this repository.

\subsubsection{Data}
\label{sec-3-4-1}
In order to use the training materials \emph{as is}, you'll need to
prepopulate a fresh database instance (i.e., an instance that only
contains nodes listed in \href{initial-data.cql}{initial-data.sql}) with the data shown
below.

If you need to add this data to many different Neo4j instances,
you can create a \texttt{.cql} script (or simply extend \href{initial-data.cql}{initial-data.sql})
and load it as described in Section \ref{sec-2-2} above.

\paragraph{Features}
\label{sec-3-4-1-1}
\begin{center}
\begin{tabular}{lll}
\hline
\textbf{name} & \textbf{description} & \textbf{type}\\
\hline
About & What is the current SpeechAct about? & atomic\\
ChildGender & Stores the gender of the current user. & atomic\\
ChildName & Stores the name of the current user. & atomic\\
CurrentGame & Stores the game that the agent and the user are currently playing. & atomic\\
Encounter & Is this the first encounter between the agent and the current user & atomic\\
 & or have they met before? & \\
Familiarity & Is the agent familiar with the current user? & atomic\\
GameQuiz & Is this the first time the agent and the current user & atomic\\
 & are playing the quiz game or have they played it before? & \\
SpeechAct & Type of utterance to perform (e.g. greeting, request) & atomic\\
\hline
\end{tabular}
\end{center}

\paragraph{Values}
\label{sec-3-4-1-2}
\begin{center}
\begin{tabular}{l}
\hline
\textbf{name}\\
\hline
Emilia\\
Marco\\
answer\\
answerRetry\\
apologize\\
closing\\
dance\\
female\\
first\\
fun\\
greeting\\
imitation\\
male\\
no\\
notfirst\\
play\\
quiz\\
request\\
underspecified\\
unknown\\
yes\\
\hline
\end{tabular}
\end{center}

Note that if you use ODE to populate the DB manually, you do
\emph{not} need to create the \texttt{underspecified} value yourself:
\href{initial-data.cql}{initial-data.sql} already contains an appropriate Cypher query for
adding this node. Just make sure you load it as described in
Section \ref{sec-2-2} above \emph{before} creating any features.

\paragraph{Associations between features and values}
\label{sec-3-4-1-3}
\begin{center}
\begin{tabular}{ll}
\hline
\textbf{Feature} & \textbf{Permitted values}\\
\hline
About & underspecified, fun, play, answerRetry, answer\\
ChildGender & underspecified, unknown, female, male\\
ChildName & underspecified, unknown, Marco, Emilia\\
CurrentGame & underspecified, quiz, imitation, dance\\
Encounter & underspecified, notfirst, first\\
Familiarity & underspecified, no, yes\\
GameQuiz & underspecified, notfirst, first\\
SpeechAct & underspecified, closing, apologize, request, greeting\\
\hline
\end{tabular}
\end{center}

Note that if you use ODE to populate the DB manually, you do
\emph{not} need to add the \texttt{underspecified} value to the list of
permitted values for each feature yourself: For each atomic
feature you create, ODE automatically sets up a relationship
between the feature in question and the \texttt{underspecified} value.

\paragraph{Rules}
\label{sec-3-4-1-4}
\begin{center}
\begin{tabular}{llll}
\hline
\textbf{name} & \textbf{description} & \textbf{LHS} & \textbf{RHS}\\
\hline
@firstEncounter & Agent meets someone for the first time. & (1) & (2)\\
\hline
\end{tabular}
\end{center}

\subparagraph{(1) LHS}
\label{sec-3-4-1-4-1}
\begin{center}
\begin{tabular}{ll}
\hline
\textbf{Feature} & \textbf{Value}\\
\hline
SpeechAct & greeting\\
Encounter & first\\
Familiarity & no\\
\hline
\end{tabular}
\end{center}

\subparagraph{(2) RHS}
\label{sec-3-4-1-4-2}
\begin{itemize}
\item Group 1:

\begin{center}
\begin{tabular}{lll}
\hline
\textbf{Slot 1} & \textbf{Slot 2} & \textbf{Slot 3}\\
\hline
Hi, & how are you? & I am Nao.\\
Hey, &  & My name is Nao.\\
Hello, &  & \\
\hline
\end{tabular}
\end{center}

\item Group 2:

\begin{center}
\begin{tabular}{ll}
\hline
\textbf{Slot 1} & \textbf{Slot 2}\\
\hline
Hola! & What's up?\\
Hey there! & Nice to meet you.\\
\hline
\end{tabular}
\end{center}
\end{itemize}

\section{Resources}
\label{sec-4}
\subsection{play!}
\label{sec-4-1}
\begin{itemize}
\item Documentation for \texttt{play-2.2.x}:
\url{https://playframework.com/documentation/2.2.x/Home}
\item Deployment:
\url{https://playframework.com/documentation/2.2.x/Production}
\item Java API for \texttt{play-2.2.x}:
\url{https://playframework.com/documentation/2.2.x/api/java/index.html}
\end{itemize}

\subsection{Neo4j}
\label{sec-4-2}
\begin{itemize}
\item Manual for Neo4j v2.1.2:
\url{http://neo4j.com/docs/2.1.2/}
\item Cypher Query Language:
\url{http://neo4j.com/docs/2.1.2/cypher-query-lang.html}
\end{itemize}

\subsection{JS + CSS Frameworks}
\label{sec-4-3}
\begin{itemize}
\item Twitter Bootstrap: \url{http://getbootstrap.com/}
\begin{itemize}
\item CSS: \url{http://getbootstrap.com/css/}
\item UI components: \url{http://getbootstrap.com/components/}
\item JavaScript: \url{http://getbootstrap.com/javascript/}
\end{itemize}
\item jQuery: \url{https://api.jquery.com/}
\item jQuery UI: \url{http://api.jqueryui.com/}
\item Backbone.js: \url{http://backbonejs.org/}
\item Underscore.js: \url{http://underscorejs.org/}
\end{itemize}

\subsection{ODE}
\label{sec-4-4}
In addition to this README, you can consult the following resources
for in-depth information about ODE:

\begin{itemize}
\item \href{http://www.dfki.de/lt/publication_show.php?id=7689}{``A System for Rapid Development of Large-Scale Rule Bases for Template-Based NLG for Conversational Agents''} (Krones 2014) (\href{http://www.dfki.de/lt/bibtex.php?id=7689}{BibTeX}, \href{http://www.dfki.de/web/forschung/iwi/publikationen/renameFileForDownload?filename=thesis-krones-final.pdf&file_id=uploads_2404}{PDF})
\begin{itemize}
\item Part III (chapters 6-7): System Architecture + Technologies
\item Part IV (chapters 9-10), Appendix A: Data Models
\item Part V (chapters 11-15): User-Facing Functionality
\item Part VII (chapter 21): Future Work
\item Appendix B: Algorithms
\end{itemize}
\item Documentation generated with Doxygen
\begin{itemize}
\item Searchable lists of packages, classes, files
\item Alphabetical index of classes
\item Textual and graphical class hierarchy
\item Alphabetical index of class members (with links to classes to
which they belong)
\item Collaboration diagrams for individual classes
\item Call and caller graphs for member functions
\item Ability to jump to definitions of class members
\end{itemize}
\item \texttt{git} commit messages associated with this repository
\end{itemize}

\subsection{Other}
\label{sec-4-5}
\begin{itemize}
\item Doxygen manual:
\url{http://www.stack.nl/~dimitri/doxygen/manual/index.html}
\item Git:
\begin{itemize}
\item \href{http://git-scm.com/book}{Pro Git}
\item \href{http://gitref.org/}{Git Reference}
\item \href{http://jonas.nitro.dk/git/quick-reference.html}{Git Quick Reference}
\item \href{http://www.ndpsoftware.com/git-cheatsheet.html}{Git Cheatsheet}
\item \href{https://www.codeschool.com/courses/try-git}{tryGit}
\end{itemize}
\end{itemize}

\section{Contact Information}
\label{sec-5}
Original author: Tim Krones (t.krones@gmx.net)
% Emacs 24.4.1 (Org mode 8.2.10)
\end{document}
